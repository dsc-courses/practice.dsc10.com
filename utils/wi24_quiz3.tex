\documentclass[twoside,12pt]{article}

\usepackage{dsctemplate}
\usepackage[margin=0.9in]{geometry}
\usepackage{amsmath}
\usepackage{amssymb,amsthm}
\usepackage{fancyhdr}
\usepackage{nicefrac}
\usepackage{minted}
\usetikzlibrary{quotes,angles,positioning,arrows.meta}
\usetikzlibrary{calc}
\usepackage{enumitem}
\usepackage{fancyvrb}
\usepackage{dirtytalk}
\usepackage{comment}

\DefineVerbatimEnvironment{verbatim}{Verbatim}{xleftmargin=.5in}

% configuration
% ------------------------------------------------------------------------------
% https://www.overleaf.com/project/651e1f6af0ac68cded0e43cf
% control whether solutions are shown or hidden
\showsolntrue

% page headers only on odd pages
\pagestyle{fancy}
\fancyhead{}
\fancyhead[RO]{PID: \rule{3in}{.5pt}}
\renewcommand{\headrulewidth}{0pt}
% \pagenumbering{gobble}

% ------------------------------------------------------------------------------

\begin{document}


\thispagestyle{empty}

\vspace{-5.5in}

\pstitle{%
    Quiz 3
}{DSC 10, Winter 2024}

\vspace{-.3in}

\begin{tabular}{rl}
    Full Name: & \inlineresponsebox[4in]{Solutions}\\
    PID: & \inlineresponsebox[4in]{A12345678}\vspace{.1in}\\
    Discussion: & \bubble{A (5PM)} \bubble{B (6PM)} \bubble{C (7PM)} \vspace{.3in} \\ 
\end{tabular}

\vspace{.1in}

\hline

\vspace{.1in}

\textbf{Instructions:}
    \begin{itemize}
        \item This quiz consists of 5 questions. You have a total of 20 minutes to complete it.
        \item Please write \textbf{clearly} in the provided answer boxes; we will not grade work that appears elsewhere. Completely fill in bubbles and square boxes; if we cannot tell which option(s) you selected, you may lose points.
        
            \bubble{A bubble means that you should only \textbf{select one choice}.}
            
            \squarebubble{A square box means you should \textbf{select all that apply}.}
        \item If your answer is a string, make sure to put it in quotes. If your answer is a float, make sure to include a decimal point.
        \item No aids are allowed (no notes, no calculators, and no computers).
    \end{itemize}

\vspace{.1in}

\hline

\vspace{2in}

\noindent By signing below, you are agreeing that you will behave honestly and fairly during
and after this quiz. 

\begin{tabular}{rl}
    \: \: \: \: \: Signature: & \inlineresponsebox[4in]{}\\
\end{tabular}

\vfill

\begin{center}
{\huge Version A} \vspace{.2in}

Please do not open your quiz until instructed to do so.

\end{center}

\newpage

\begin{center}
    \includegraphics[width=2in]{quiz_images/plane.jpg}
\end{center}

\noindent In the DataFrame \texttt{flights}, each row corresponds to a recent flight. The columns are:

\begin{itemize}
    \item \texttt{flight\_num} (\texttt{str}): The unique code of the flight, consisting of a \textbf{two-character airline designator} followed by 1 to 4 digits (e.g. \texttt{"UA1989"}).
    \item \texttt{airline} (\texttt{str}): The airline name (e.g. \texttt{"United"}).
    \item \texttt{departure} (\texttt{str}): The code for the airport from which the flight departs (e.g. \texttt{"SAN"}). 
    \item \texttt{arrival} (\texttt{str}): The code for the airport at which the flight arrives (e.g. \texttt{"LAX"}).
\end{itemize}

\begin{probset}

\begin{prob}
Suppose the function \texttt{first\_two} takes as input a string and returns a string with the first two characters only. For example \texttt{first\_two("panda")} is \texttt{"pa"}. Using this function, write an expression that evaluates to a Series containing the \textbf{two-character airline designator} for each flight in \texttt{flights}.

\begin{responsebox}{0.45in}
    \texttt{flights.get("flight\_num").apply(first\_two)}
\end{responsebox}
\end{prob}

\begin{prob}

\begin{subprobset}

\begin{subprob}

Fill in the blanks below so that \texttt{grouped} is a DataFrame showing how many flights on each airline departed from each airport.

 \begin{verbatim}
grouped = flights.groupby(___(x)___).___(y)___.get(['flight_num'])
\end{verbatim}   

\texttt{(x)}: \inlineresponsebox[3.8in]{\texttt{["departure", "airline"]} or \texttt{["airline", "departure"]}}{}
\texttt{(y)}: \inlineresponsebox[1.5in]{\texttt{count()}}{} 

\end{subprob}

\begin{subprob}
Suppose the expression \texttt{grouped.shape[0] == flights.shape[0]} evaluates to \texttt{True}. \\Select all true statements below.

\squarebubble{No two flights in \texttt{flights} were on the same airline.} 

\squarebubble{No two flights in \texttt{flights} had the same departure and arrival airports.} 

\correctsquarebubble{Among all flights in \texttt{flights} that were on the same airline, no two had the same \\\hspace*{0.25in}departure airport.} 

\correctsquarebubble{Among all flights in \texttt{flights} from the same departure airport, no two were on the  \\\hspace*{0.25in}same airline.} 

\end{subprob}
  
\end{subprobset}

\end{prob}
\newpage
\vspace{-0.2in}
\begin{prob}

Suppose we have another DataFrame \texttt{more\_flights} which contains the same columns as \texttt{flights}, but different rows. Define \texttt{merged} as follows.

\begin{verbatim}
    merged = flights.merge(more_flights, on = "airline")
\end{verbatim}  
Suppose that in \texttt{merged}, there are 108 flights where the airline is \texttt{"United"}, and in \texttt{more\_flights}, there are 12 flights where the airline is \texttt{"United"}. If \texttt{flights} has 15 rows in total, how many of these rows are \textbf{not} for \texttt{"United"} flights? Give your answer as an integer.
\begin{center}
\inlineresponsebox[1in]{\texttt{6}}{} 
\end{center}




\end{prob}

\vspace{-0.2in}
\begin{prob}

\begin{subprobset}

\begin{subprob}
Fill in the blanks below so that \texttt{prob} evaluates to the probability that a randomly selected flight from the \texttt{flights} DataFrame arrives at \texttt{"SAN"} given that it departs from \texttt{"LAX"}.

\begin{verbatim}
san = flights.get("arrival") == "SAN"
lax = flights.get("departure") == "LAX"
prob = flights[san___(x)___lax].shape[0] / ___(y)___
\end{verbatim}

\texttt{(x)}: \inlineresponsebox[5.85in]{\texttt{\&}}{}\\
\texttt{(y)}: \inlineresponsebox[5.85in]{\texttt{flights[lax].shape[0]}}{}

\end{subprob}

\begin{subprob}

Write an expression that evaluates to the probability that a randomly selected flight from the \texttt{flights} DataFrame is \textbf{not} on the airline \texttt{"Delta"}.

\begin{responsebox}{0.5in}
\texttt{1 - flights[flights.get("airline") == "Delta"].shape[0] / flights.shape[0]} or 
\texttt{flights[flights.get("airline") != "Delta"].shape[0] / flights.shape[0]} or
\texttt{(flights.get("airline") != "Delta").mean()}
\end{responsebox}

\end{subprob}
\end{subprobset}
\end{prob}
\vspace{-0.15in}
\begin{prob}

Use the function defined below to answer the questions on the right.



\begin{minipage}[t]{2.8in}
\begin{verbatim}
def discount(duration, price):
    if duration > 6:
        price = price * 0.9
    if duration > 4:
        price = price * 0.95 
    elif duration > 3:
        price = price * 0.98
    else:
        price = price * 0.99 
    price = price - 5 
    return price
\end{verbatim}
\end{minipage} \hspace{0.25in}
\begin{minipage}[t]{3.5in}
\begin{subprobset}
    \begin{subprob}
    What is the output of \texttt{discount(7, 100)}?

\bubble{\texttt{100 * 0.9 * 0.95 * 0.98 - 5}} 

\correctbubble{\texttt{100 * 0.9 * 0.95 - 5}} 

\bubble{\texttt{100 * 0.9 - 5}} 

\bubble{\texttt{100 * 0.9}}
    \end{subprob}
    \begin{subprob}
        Find \texttt{x} and \texttt{y} such that \texttt{discount(x, y)} evaluates to \texttt{97 * 0.98 - 5}.
    \end{subprob}
    
    \texttt{x}: \inlineresponsebox[1in]{anything in \texttt{(3, 4]}}{}
\texttt{y}: \inlineresponsebox[1 in]{\texttt{97}}{}
\end{subprobset}
\end{minipage}
\end{prob}

\end{probset}

\newpage

\begin{center}
\textbf{Before submitting your quiz, make sure your PID is on the front page and on the top right corner of page 3.}
\end{center}

\end{document}

\begin{prob}

\begin{subprobset}

\begin{subprob}
    Bill is interested in predicting the class of airline tickets based on their cost. He establishes the following price ranges for different classes:

    \begin{itemize}
        \item \texttt{"Economy"}: Less than or equal to \$200
        \item \texttt{"Business"}: \$200 to \$400 (inclusive)
        \item \texttt{"First Class"}: above \$400
    \end{itemize}

    Fill in the lines of code for the following function, which takes in a ticket price and outputs the predicted class based on Bill's ranges.
    
\begin{verbatim}
def predict_plane_class(cost):
    _____(x)_____:
        return 'First Class'
    _____(y)_____:
        return 'Business'
    _____(z)_____:
        return 'Economy'
\end{verbatim}

\texttt{(x)}: \inlineresponsebox[1.75 in]{\texttt{if cost > 400}}{}
\texttt{(y)}: \inlineresponsebox[1.75 in]{\texttt{elif cost > 200}}{}
\texttt{(z)}: \inlineresponsebox[1.75 in]{\texttt{else}}{}

\end{subprob}

\newpage

\begin{subprob}
    Create a new column \texttt{"plane\_class"} in the \texttt{passengers} DataFrame using the function defined in part \textbf{a} by completing the following code:

\begin{verbatim}
passengers = passengers.___(x)___(plane_class = ___(y)___.___(z)___)
\end{verbatim}

\texttt{(x)}: \inlineresponsebox[5.85 in]{\texttt{assign}}{}
\texttt{(y)}: \inlineresponsebox[5.85 in]{\texttt{passengers.get("price")}}{}
\texttt{(z)}: \inlineresponsebox[5.85 in]{\texttt{apply(predict\_plane\_class)}}{}

\end{subprob}
    
\end{subprobset}

\end{prob}